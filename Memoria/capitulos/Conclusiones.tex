\chapter{Conclusiones y Trabajos Futuros}

\section{Relación del \acrshort{TFM} con lo aprendido en el Máster}

GIW

SSBW

CC


\section{Conclusiones}

Proyecto interesante

Curiosidad planteada en la motivacion satisfecha

No tiempo suficiente para llevar a cabo todo lo planteado. Objetivo de  Evaluación no cumplido

\section{Trabajos futuros}
Relevance feedback

Redes de citación

Intentar obtener Altmetrics de otras fuentes

\textbf{Evaluar la mejora producida al aplicar medidas bibliométricas al sistema}: como colección de datos del sistema \acrshort{RI} a desarrollar se ha decidido utilizar el conjunto de todos los trabajos de autores de la \myFaculty lo que favorece que esta evaluación se pueda llevar a cabo, aunque esta no es una tarea sencilla. La evaluación de sistemas \acrshort{RI} se suele llevar a cabo mediante colecciones de prueba donde existe un conjunto de documentos, un conjunto de consultas y un conjunto de resultados que deberían de retornar las mismas  \cite{RI_Evaluation}. Al crear un sistema nuevo con una colección de documentos no utilizada no podemos comparar los resultados obtenidos con los que se esperarían, este problema se debe a la relatividad del concepto de relevancia. Por ello las colecciones de pruebas suelen incluir valoraciones de relevancia de expertos en la materia.