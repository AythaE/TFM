\chapter{Análisis}

\section{Enfoque planteado}
Como se ha podido comprobar en el apartado previo existen numerosas propuestas para combinar la bibliometría y la \acrshort{RI}, pero estos trabajos se realizan en un ámbito más académico y muchas veces no llegan a materializarse en sistemas reales. Los principales sistemas de \acrshort{RI} para la recuperación de literatura académica integran algunas medidas básicas pero no llegan a implementar los modelos más complejos planteados desde la investigación a pesar de que sus resultados sean esperanzadores. 

Esto hace que este campo siga estando en desarrollo y resulte interesante plantear nuevos modelos a la par que testearlos.

Por ello este \acrshort{TFM} servirá para desarrollar un prototipo de modelo que combine ambas ramas de la teoría de la información y sirva para evaluar el rendimiento de estos planteamientos. El objetivo es comparar un sistema de \acrshort{RI} clásico con uno que incorpore técnicas bibliométricas intentando medir su viabilidad y potencial mejora en los resultados recuperados.

En los últimos tiempos el acceso a artículos se ha incrementado exponencialmente gracias a algunas de las plataformas descritas previamente. A su vez, la información bibliográfica que incorporan también ha aumentado sustancialmente lo que hace cada vez más plausible la implementación de este tipo de sistemas. La irrupción de las \textit{altmetrics} resulta destacable. A pesar de ser unas medidas bibliométricas bastante recientes, su capacidad para determinar la popularidad de los trabajos científicos es muy significativa.

Es especialmente interesante el enfoque de las técnica híbridas, como las planteadas en el último trabajo analizado, que permiten por un lado no divergir del modelo mental de sistema de \acrshort{RI} del usuario, un buscador textual, pero incluir las potenciales ventajas de usar otro tipo de medidas. 

El sistema propuesto utilizará como base un modelo clásico realizando una reordenación \textit{a priori} de los resultados en función de algunas medidas directas como el número de citas o el número de lectores en una plataforma (incluyendo con ello alguna \textit{altmetric}). Se plantea utilizar una realimentación inconsciente del usuario basada en considerar como relevantes los documentos del listado inicial que el usuario descargue tras leer el resumen y utilizándolos como semillas para refinar la búsqueda de manera transparente para él. Esta reordenación  \textit{a posteriori} se servirá del grafo de citación de estos documentos semilla para seleccionar los que documentos que tengan una relación fuerte.

\section{Historias de usuario}
Siguiendo los enfoques de las metodologías ágiles como la empleada, en lugar de definir los requerimientos y alcance del software mediante una especificación de requisitos de usuario o casos de uso, me he decantado por utilizar historias de usuario.

El usuario que se mencionará en estas historias así como el del sistema final será un usuario relacionado con el mundo científico y que esté familiarizado con los conceptos típicos como las medidas bibliométricas empleadas.

En la siguiente lista se recogen las historias de usuario con las que se ha trabajado

\begin{itemize}
	\item \textbf{Seleccionar el método de ordenación a \textit{priori} de los resultados de búsqueda}: El usuario tendrá que poder elegir entre varios criterios de ordenación de resultados basados en medidas bibliométricas directas. Los criterios concretos dependen de la disponibilidad de datos bibliográficos por lo que se definirán más adelante.
	
	\item \textbf{Seleccionar el método de ordenación a \textit{posteriori} de los resultados}: El usuario ha de tener la posibilidad de seleccionar entre varios métodos de ordenación de los resultados en base a su interacción con el sistema.

	\item \textbf{Realizar búsquedas de autores}: El usuario podrá ejecutar búsquedas de autores entre el colección de los mismos. Por motivos de comodidad y familiaridad se utilizarán como autores aquellos pertenecientes a la \myFaculty de la \myUni.
	
	\item \textbf{Realizar búsquedas de artículos}: El usuario deberá tener la posibilidad de hacer búsquedas de artículos en el sistema. Al igual que en la historia previa el conjunto de artículos estará constituido por los trabajos de los autores mencionados previamente.
	
	\item \textbf{Desplegar una vista detallada de un artículo}: El usuario dispondrá de la posibilidad de desplegar una vista detallada de cada uno de los artículos devueltos en la búsqueda. Dicha vista tiene como objetivo mostrar una información básica sobre los artículos que permita discernir su contenido y relevancia, así como permitir enlazar a otras fuentes de informaron más detalladas sobre el mismo. Contendrá al menos los siguientes datos:
	\begin{itemize}
		\item[\textendash] Título
		\item[\textendash] Autores
		\item[\textendash] Resumen
		\item[\textendash] Información bibliográfica
		\item[\textendash] Palabras clave
	\end{itemize}
	
	\item \textbf{Desplegar una vista detallada de un autor}:  El usuario dispondrá de la posibilidad de desplegar una vista detallada de cada uno de los autores devueltos en la búsqueda. Al igual que en el caso anterior, ha de contener un extracto de información sobre el autor, así como enlaces a otras fuentes de datos o perfiles con más datos. Contendrá al menos los siguientes datos:
	\begin{itemize}
		\item[\textendash] Nombre
		\item[\textendash] Apellidos
		\item[\textendash] Departamento / Grupo de investigación
		\item[\textendash] Información bibliográfica

	\end{itemize}
	
\end{itemize}

