\chapter{Introducción}
\section{Motivación}

A día de hoy usamos constantemente buscadores: web como Google o Bing, de ficheros como los integrados en todos los sistemas de ficheros modernos o de contenido como puede ser una búsqueda de algún término en un \acrshort{PDF} como este. Todo buscador es conocido desde un punto de vista más técnico como un \textbf{Sistemas de \acrfull{RI}}. 

Tradicionalmente estos sistemas realizan una búsqueda por contenido, si buscamos una palabra o término de búsqueda concreto en Google, este retorna páginas relevantes que contentan dicha palabra. 

Esto también funciona así para la recuperación de artículos científicos, pero en este caso los artículos científicos disponen de algunas medidas asociadas como el número de citas que pueden ser interesantes para determinar la relevancia de un artículo. Se puede entender que si un par de artículos contienen un término de búsqueda, él que sea más citado parece, a \textit{priori}, más relevante, ya que la propia comunidad científica lo menciona con mayor frecuencia. Dichas métricas asociadas a la literatura científica se conocen como \textbf{medidas bibliométricas}.

Este proyecto pretende explorar las posibilidades de estas medidas para mejorar los procesos de recuperación de información.

Durante la asignatura del máster \acrfull{GIW}, se vieron algunas pinceladas de las herramientas empleadas para llevar a cabo sistemas de \acrlong{RI}, así como su base teórica. Esto me llamó realmente la atención, ya que todos utilizamos diariamente sistemas de búsqueda, pero no tenía ni idea de como se podía implementar uno. A pesar de haber desarrollado un pequeño sistema como parte de sus prácticas, me quedé con la ganas de ver un sistema "más real", desarrollado con herramientas más potentes. Este es el principal motivo por el que me decanté por la realización de este proyecto, a nivel personal me gustaría poder llenar esta curiosidad con el desarrollo del presente trabajo.

\section{Objetivos}

En este apartado recogeré de manera sintetizada los objetivos del proyecto, lo cual ayudará a comprender la funcionalidad del sistema a desarrollar así como definir su alcance. El objetivo principal es \textbf{desarrollar un Sistema de \acrlong{RI} que incorpore medidas bibliométricas como mejora a la recuperación clásica}. Junto a este objetivo también tenemos 

%\section{Objetivos generales}
\begin{itemize}

	
	\item \textbf{Descomponer el sistema de \acrshort{RI}}: Observando los diversos sistemas reales de recuperación de información científica que he analizado, como Google Scholar o Scopus, la gran mayoría dividen la búsqueda en búsqueda de autores y de artículos en sí. Ya que estos son dos tipos de datos diferenciados (aunque relacionados profundamente) e intuitivamente, se comprende que un sistema de RI funcionará mejor si los datos del mismo son homogéneos. Por ello el sistema a desarrollar dispondrá de dos partes una búsqueda de autores y otra de artículos.
	
	\item \textbf{Desarrollar un sistema que sea usable}: Muchos de los enfoques que he visto durante mi proceso de documentación no pasan de modelos teóricos, prototipos o sistemas en los que la usabilidad y la orientación al usuario brillan por su ausencia. Aunque este no sea el objetivo principal del sistema, me parece muy importante que se tenga en cuenta al usuario durante todo el proceso de desarrollo, ya que un sistema puede ser increíble pero si los usuarios no lo entienden o no le saben sacar partido, no sirve de nada. Por ello, pretendo diseñar una interfaz de usuario que sea simple de usar, empleando metáforas y componentes ampliamente conocidos por los usuarios.

\end{itemize}


\section{Organización de la memoria}

Esta memoria detallará el desarrollo del proyecto desde su comienzo hasta su conclusión. Todos los artefactos generados durante el proyecto se encuentran recogidos en el repositorio \textit{Github} \url{https://github.com/AythaE/TFM}.

\newpage
Este documento cuenta con los siguientes apartados:
\begin{itemize}
	\item El siguiente capítulo versa sobre los aspectos de la \textbf{planificación} del proyecto, desde una perspectiva temporal, económica, de recursos y metodológica.
	\item Tras esto se definirá el \textbf{contexto} del trabajo, dando antecedentes, definiendo los principales conceptos teóricos y analizando el estado del arte.
	\item En el capítulo \textbf{análisis} se planteará el proyecto a desarrollar, ayudándose de historias de usuario para definir la funcionalidad del sistema a crear.
	\item Este planteamiento se refinará y detallará en el \textbf{diseño}, centrándose en los datos y arquitectura del sistema.
	\item El grueso de la memoria está constituido por el \textbf{desarrollo}, donde se han apuntado los principales aspectos del mismo, siguiendo una estructura de \textit{sprints}, así como la aclaración de la arquitectura del sistema final.
	\item Para finalizar el contenido principal, el apartado de \textbf{conclusiones y trabajos futuros} recoge algunas reflexiones personales y sobre los resultados del proyecto, incluyendo algunos posibles caminos futuros de ampliación del mismo.
	\item Tras esto se pueden encontrar la \textbf{bibliografía}, así como los anexos \textbf{glosario}, \textbf{lista de acrónimos}, un \textbf{manual de usuario} con una breve explicación de las pantallas y su uso, así como un \textbf{manual técnico} donde se detalla la arquitectura final con instrucciones de instalación y despliegue.
\end{itemize}


