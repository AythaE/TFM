\chapter{Introducción}
\section{Motivación}


Durante la asignatura del máster \acrfull{GIW} se vieron algunas pinceladas de las herramientas empleadas para llevar a cabo sistemas de \acrfull{RI} así como su base teórica. Esto me llamó realmente la atención ya que todos utilizamos diariamente sistemas de búsqueda, pero no tenía ni idea de como se podía implementar uno. A pesar de haber desarrollado un pequeño sistema como parte de sus prácticas me quedé con la ganas de ver un sistema más "real" desarrollado con herramientas más potentes. Este es el principal motivo por el que me decanté a la realización de este proyecto, a nivel personal me gustaría poder llenar esta curiosidad con desarrollo del presente trabajo.

\section{Objetivos}

En este apartado recogeré de manera sintetizada los objetivos del proyecto, lo cual ayudará a comprender la funcionalidad del sistema a desarrollar así como definir su alcance. El objetivo principal es \textbf{comprobar los resultados de la aplicación de medidas bibliométricas a un sistema de \acrfull{RI}}. Junto a este objetivo también tenemos 

%\section{Objetivos generales}
\begin{itemize}
	\item \textbf{Evaluar la mejora producida al aplicar medidas bibliométricas al sistema}: como colección de datos del sistema \acrshort{RI} a desarrollar se ha decidido utilizar el conjunto de todos los trabajos de autores de la \myFaculty lo que favorece que esta evaluación se pueda llevar a cabo, aunque esta no es una tarea sencilla. La evaluación de sistemas \acrshort{RI} se suele llevar a cabo mediante colecciones de prueba donde existe un conjunto de documentos, un conjunto de consultas y un conjunto de resultados que deberían de retornar las mismas  \cite{RI_Evaluation}. Al crear un sistema nuevo con una colección de documentos no utilizada no podemos comparar los resultados obtenidos con los que se esperarían, este problema se debe a la relatividad del concepto de relevancia. Por ello las colecciones de pruebas suelen incluir valoraciones de relevancia de expertos en la materia.
	\item \textbf{Desarrollar un sistema que sea usable}: Muchos de los enfoques que he visto durante mi proceso de documentación no pasan de modelos teóricos, prototipos o sistemas en los que la usabilidad y la orientación al usuario brillan por su ausencia. Aunque este no sea el punto objetivo principal del sistema me parece muy importante que se tenga en cuenta al usuario durante todo el proceso de desarrollo, ya que un sistema puede ser increíble pero si los usuarios no lo entienden o no le saben sacar partido no sirve de nada. Por ello pretendo diseñar una interfaz de usuario que sea simple de usar y emplee metáforas y componentes ampliamente conocidos por los usuarios.

\end{itemize}


\section{Organización de la memoria}


