\chapter{Técnicas y herramientas [Aún en progreso]}
\label{ch:herramientas}

Comentar en más detalle todas las herramientas utilizadas con enlaces de cada una.

\begin{itemize}
	\item \textbf{Debian}: \acrfull{SO}.
	\item \textbf{Python}: Lenguaje de programación usado en las primeras fases del proyecto y en servidor \gls{backend} \glsrefentry{backend}.
	\item \textbf{JavaScript}: Lenguaje de programación interpretado en el que se ha escrito el \gls{frontend} \glsrefentry{frontend}.
	\item \textbf{Elasticsearch}: Servidor de búsqueda.
	\item \textbf{MongoDB}: Base de datos NoSQL.
	\item \textbf{React}:  \Gls{framework} para el desarrollo de interfaces de usuario.
	\item \textbf{Searchkit}: \Gls{framework} que incluye un conjunto de componentes React para la comunicación con Elasticsearch.
	\item \textbf{TeXstudio}: Entorno integrado de escritura en \LaTeX{} utilizado para generación de la documentación.
	\item \textbf{Docker}: Software de virtualización para basado en contenedores. Permite gestionar de forma simple la gestión y despliegue de una infraestructura software.
	\item \textbf{Visual Studio Code}: Editor de código creado por Microsoft utilizado para toda la programación del proyecto.
	\item Git
	\item GitHub
	\item Pandas
	\item PyMongo
	\item scopus-api
	\item Beautiful Soup
	\item Matplotlib
	\item MaterialUI
	\item elasticsearch-py
	\item Star UML
	\item Cerebro
	\item Flask
	\item Gimp
	\item Nginx
	\item uWSGI
\end{itemize}