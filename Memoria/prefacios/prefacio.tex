\chapter*{}
%\thispagestyle{empty}
%\cleardoublepage

%\thispagestyle{empty}

%\input{portada/portada_2}



\cleardoublepage
\thispagestyle{empty}

\begin{center}
{\large\bfseries \myTitle}\\
\end{center}
\begin{center}
\myName\\
\end{center}

%\vspace{0.7cm}
\noindent{\textbf{Palabras clave}: Recuperación de información, Bibliometría, \acrlong{ES}}\\

\vspace{0.7cm}
\noindent{\textbf{Resumen}}\\

Todos utilizamos motores de búsqueda en nuestra vida diaria y cada vez somos más dependientes de los mismos, debido al exponencial incremento de información que se genera diariamente en los últimos tiempos. Esto hace cada vez más importante la mejora de los sistemas de búsqueda, para que sean capaces de priorizar y ayudarnos a encontrar la información que necesitamos.

En el mundo científico también ocurre esto, un investigador necesita la ayuda de algún sistema de recuperación de información para poder mantenerse al día en su rama de investigación. Por ello este proyecto propone un modelo alternativo de sistema de búsqueda, que tenga en cuenta medidas bibliométricas características de los artículos científicos, como el número de citas o el índice h, para mejorar la recuperación.

En concreto, el sistema planteado combinará la información bibliométrica con los resultados de búsquedas por contenido tradicionales de distintos modos. Este sistema se ha desarrollado como una aplicación web distribuida, que se servirá de una interfaz gráfica de usuario para realizar búsquedas por autores o artículos y comparar los resultados obtenidos con las distintas combinaciones.

Para crear dicho sistema se ha empleado una metodología de desarrollo ágil que permite desarrollar de forma rápida e iterativa.


\cleardoublepage


\thispagestyle{empty}
%
%
\begin{center}
{\large\bfseries Development of a search engine prototype that incorporate bibliometric techniques to improve retrieval}\\
\end{center}
\begin{center}
\myName\\
\end{center}

%\vspace{0.7cm}
\noindent{\textbf{Keywords}: Information retrieval, Bibliometry, \acrlong{ES}}\\

\vspace{0.7cm}
\noindent{\textbf{Abstract}}\\

Everyone use search engine in our daily life and we are becoming more and more dependent from them, due to the exponential increase of information generated on a daily basis nowadays. This make increasingly important the improve of search systems, they need to be able to prioritize and help us find the information we need.

The same happens in the scientific world, a researcher need the help of an information retrieval system in order to be up to date in his investigation field. That why this project aims to achieve an alternative information retrieval system model, taking into account the bibliometric measures characteristic from scientific papers, like the cites number or the h index, to improve the retrieval.

Specifically, the proposed system will combine bibliometric information with traditional content search results in different ways. This system has been developed as a distributed web application, that by the use of a graphic user interface to perform search by authors or papers and compare the results obtained with the different combinations.

To create the mentioned system an agile development methodology has been used, it allows a quick and iterative development.
%
\chapter*{}
\thispagestyle{empty}

\noindent\rule[-1ex]{\textwidth}{2pt}\\[4.5ex]

Yo, \textbf{\myName}, alumno de la titulación \myDegree de la \textbf{\myFaculty de la \myUni}, con DNI 70918176E, autorizo la
ubicación de la siguiente copia de mi Trabajo Fin de Máster en la biblioteca del centro para que pueda ser
consultada por las personas que lo deseen.

\vspace{6cm}

\noindent Fdo: \myName

\vspace{2cm}

\begin{flushright}
\myLocation a \myTime.
\end{flushright}


\chapter*{}
\thispagestyle{empty}

\noindent\rule[-1ex]{\textwidth}{2pt}\\[4.5ex]

D. \textbf{\myProf}, Profesor del Área de Ciencias de la Computación e Inteligencia Artificial del \myDepartment
 de la Universidad de Granada.

\vspace{0.5cm}

\textbf{Informa:}

\vspace{0.5cm}

Que el presente trabajo, titulado \textit{\textbf{\myTitle}},
ha sido realizado bajo su supervisión por \textbf{\myName}, y autorizo la defensa de dicho trabajo ante el tribunal
que corresponda.

\vspace{0.5cm}

Y para que conste, expiden y firman el presente informe en Granada a \myTime.

\vspace{1cm}

\textbf{El director:}

\vspace{5cm}

\noindent \textbf{\myProf }
%
%\chapter*{Agradecimientos}
%\thispagestyle{empty}
%
%       \vspace{1cm}


%Poner aquí agradecimientos...

