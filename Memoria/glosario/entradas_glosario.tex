% Glosario
\newglossaryentry{termIndex}
{
	name=términos de indexación,
	description={concepto asociado a una serie de palabras donde el concepto es definido o discutido}
}
\newglossaryentry{recomenderSystem}{name={sistema de recomendación},description={Tipo de sistema de \acrshort{RI} en el que el usuario no expresa directamente su necesidad de información, si no que se le muestran items "similares" a los que ya ha consultado}}
\newglossaryentry{Bayes}{name={Teorema de Bayes},description={Teorema perteneciente a la teoria de probabilidades que permite calcular la probabilidad condicional de dos eventos \textit{A} y \textit{B} en base su probabilidad condicional inversa y su probabilidad marginal \[P(A|B) = \frac{P(B|A)P(A)}{P(B)}\]
Donde $P(A|B)$ es la probabilidad condicional de que dado un evento \textit{B} se produzca \textit{A}, $P(A)$ la probabilidad márginal o incondicional de que suceda \textit{A}, $P(B|A)$ la probabilidad condicional de que dado A suceda B y $P(B)$ la probabilidad marginal de \textit{B}}}

% Acronimos
\newacronym{PDF}{PDF}{Portable Document Format}
\newacronym{RI}{RI}{Recuperación de Información}
\newacronym{WoS}{WoS}{Web of Science}
\newacronym{FECYT}{FECYT}{Fundación Española para la Ciencia y la Tecnología}
\newacronym{UGR}{UGR}{Universidad de Granada}
\newacronym{BIR}{BIR}{Bibliometric-enhanced Information Retrieval}
\newacronym{Mr.DLib}{Mr.DLib}{\underline{M}achine-\underline{r}eadable \underline{D}igital \underline{Lib}rary}
\newacronym{TFM}{TFM}{Trabajo de Fin de Master}