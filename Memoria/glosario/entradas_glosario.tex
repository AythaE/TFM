% Glosario
\newglossaryentry{termIndex}
{
	name=términos de indexación,
	description={concepto asociado a una serie de palabras donde el concepto es definido o discutido}
}
\newglossaryentry{recomenderSystem}{name={sistema de recomendación},description={Tipo de sistema de \acrshort{RI} en el que el usuario no expresa directamente su necesidad de información, si no que se le muestran items "similares" a los que ya ha consultado}}

\newglossaryentry{backend}{name={\textit{backend}},description={Parte de un sistema informático encargada del procesamiento o almacenamiento de información \cite{backend_frontend}}}

\newglossaryentry{frontend}{name={\textit{frontend}},description={Parte de un sistema informático encargada de la interacción con el usuario \cite{backend_frontend}}}

\newglossaryentry{framework}{name={\textit{framework}},description={Abstracción que provee un entorno reutilizable y genérico que puede ser utilizado para facilitar el desarrollo software \cite{framework}}}

\newglossaryentry{webscraping}{name={\textit{web scrapping}},description={Extracción de datos de una página web de forma automática manejando directamente los ficheros HTML de las páginas web \cite{webscrapping}}}

\newglossaryentry{Scrum}{name={Scrum},description={Metodología ágil de desarrollo de software basa en la descomposición del proyecto en sprints en los que definen los objetivos en lugar de como hacer cada paso del proyecto. Existen tres roles en un proyecto Scrum: 
\begin{itemize}
	\item Equipo de desarrollo: formado por varios integrantes habitualmente
	\item ScrumMaster: parte del equipo de desarrollo pero con un rol especial que puede ser entendido como el "entrenador" del equipo
	\item Product owner: representa el cliente o los usuarios finales que son los que finalmente usaran el software desarrollado.
\end{itemize}
Cada uno de los sprint cuentan con un conjunto de tareas definidas a cumplir conocidas como sprint backlog y se suele hacer uso de tableros para seguir el progreso de cada una de las mismas. \cite{scrum}}}

\newglossaryentry{Bayes}{name={Teorema de Bayes},description={Teorema perteneciente a la teoria de probabilidades que permite calcular la probabilidad condicional de dos eventos \textit{A} y \textit{B} en base su probabilidad condicional inversa y su probabilidad marginal \[P(A|B) = \frac{P(B|A)P(A)}{P(B)}\]
Donde $P(A|B)$ es la probabilidad condicional de que dado un evento \textit{B} se produzca \textit{A}, $P(A)$ la probabilidad márginal o incondicional de que suceda \textit{A}, $P(B|A)$ la probabilidad condicional de que dado A suceda B y $P(B)$ la probabilidad marginal de \textit{B}}}

% Acronimos
\newacronym{PDF}{PDF}{\textit{Portable Document Format}}
\newacronym{RI}{RI}{Recuperación de Información}
\newacronym{WoS}{WoS}{\textit{}Web of Science}
\newacronym{FECYT}{FECYT}{Fundación Española para la Ciencia y la Tecnología}
\newacronym{UGR}{UGR}{Universidad de Granada}
\newacronym{BIR}{BIR}{\textit{Bibliometric-enhanced Information Retrieval}}
\newacronym{Mr.DLib}{Mr.DLib}{\textit{\underline{M}achine-\underline{r}eadable \underline{D}igital \underline{Lib}rary}}
\newacronym{TFM}{TFM}{Trabajo de Fin de Máster}
\newacronym{TFG}{TFG}{Trabajo de Fin de Grado}
\newacronym{GIW}{GIW}{Gestión de Información en la Web}
\newacronym{SO}{SO}{Sistema Operativo}
\newacronym{ETSIIT}{\myFacultyShort}{\myFaculty}
\newacronym{API}{API}{\textit{Application Programming Interface}}
\newacronym{ES}{ES}{\textit{Elasticsearch}}
\newacronym{BD}{BD}{Base de Datos}
\newacronym{REST}{REST}{\textit{Representational State Transfer}}
\newacronym{HTTP}{HTTP}{\textit{HyperText Transfer Protocol}}
\newacronym{XML}{XML}{\textit{E\underline{x}tensible Markup Language}}
\newacronym{VPN}{VPN}{\textit{Virtual Private Network}}
\newacronym{URL}{URL}{\textit{Universal Resource Locator}}
\newacronym{DBaaS}{DBaaS}{\textit{DataBase as a Service}}